\documentclass[11pt]{article}
\usepackage[includeheadfoot, top=1.0in, bottom=1.0in, hmargin=1.0in]{geometry}
\usepackage[utf8]{inputenc}
\usepackage{fancyhdr}
\usepackage{url}
\pagestyle{fancy}
\usepackage{setspace}
\usepackage{tabularx}
\usepackage{graphicx}
\usepackage{caption}
\usepackage{subcaption}
\usepackage{hyperref}
\usepackage{multicol}
\usepackage{amsmath}
\usepackage{enumitem}

\usepackage{hyperref}
\hypersetup{
    colorlinks=true,
    linkcolor=blue,
    filecolor=magenta,      
    urlcolor=blue,
}


\lhead{Astronomy Lab II}
\rhead{Fall 2024}
\lfoot{Porter}
\rfoot{Wed 6-9PM}
\cfoot{\thepage}

\begin{document}

\begin{center}
\huge{Final Projects}\\ \medskip \Large{Presentations and Reflections due December 4, 2024}
\end{center}

\section{Overview}
For our last class meeting, in place of an exam, each of you will present a project on an astronomy topic of your choice.  I have provided a list of suggested projects and topics here, but you are welcome to come up with a topic of your own.  All projects must be approved by Lori by \textbf{November 13}, and no two topics may be the same, so they will be approved on a first-come, first-serve basis.  This is your opportunity to explore something that intrigues \textit{you} about astronomy and to share that in a creative way with us and your classmates -- so have fun!  Our last class will be dedicated to project preparation. I am available by email or appointment if you would like to get feedback or practice your presentation.

I HIGHLY encourage you to use your own respective hobbies, interestes, or field of study. Powerpoints are perfectly fine, but I would love to see you get creative and show me how you can connect astronomy to your major! For example, a political science major may find doing a final project on space policy interesting, while someone in finance may investigate funding, or a humanities major could be interested in representing space and science through art, and showing that the two are not mutually exclusive. \textbf{If you're particularly stuck, I am happy to help you come up with some ideas after you've done some brainstorming of your own.}

You will have at LEAST Lab 9, November 13, to work on and receive feedback on your project, though I encourage you to start earlier if you finish the class labs early.

\section{Guidelines}
\subsection*{Preparation}
\begin{itemize}[noitemsep]
    \item Select a project topic and medium and have your project instructor-approved by \textbf{November 13}.
    \item Prepare a 10-minute presentation about your project to deliver in class on \textbf{Wednesday, December 4}. If you'll be using slides, you should submit your presentation slides by midnight on \textbf{Tuesday, December 3} (i.e., the night before the presentations). Where appropriate, you should also include references (to, e.g., research papers, popular science articles, websites, books, etc.) in your slides; no special formatting or citation style is needed.
    \item Prepare a reflection on your project to submit on Courseworks by \textbf{December 4} as well.
\end{itemize}
\subsection*{Presentations}
\begin{itemize}[noitemsep]
    \item You are welcome to get feedback from others, but each individual should be doing their own project. I encourage you to work together when allowed during lab time to discuss your ideas, and you will be peer-reviewing each other as well. 
    \item Presentations should be  \textbf{around 10 minutes in length}
    \item Each presentation will be followed by a \textbf{5 minute question period}
    \item All presentations should include a description of the science underlying your chosen topic as well as the limitations of our collective knowledge on this topic (see rubric under ``Grading" section).
    \item For research project presentations, you should choose a topic that we have not covered \textbf{in detail} in class. You should be learning something new! For your presentation, you may use any combination of slides and/or whiteboard.
    \item For non-research projects (i.e., projects that creatively interpret a topic we've covered in class), you should also include in your presentation:
    \begin{itemize}
        \item A description and presentation of the project itself
        \item Why you chose this specific medium and how you are using it to convey key information to your audience.
    \end{itemize}
    \item Come ready to ask questions during and after each presentation; questions will count for a participation grade.  Any type of question is welcome (e.g., asking the presenter to clarify a statement, asking the presenter for more background information, asking the presenter hypothetical questions based on relevant scenarios) -- remember, there's no such thing as a bad question!
\end{itemize}

\subsection*{Reflections}
\begin{itemize}[noitemsep]
    \item Reflections should be \textbf{at least 1 page in length}
    \item Each reflection should include the following:
    \begin{itemize}
        \item A description of the underlying science concept and how this concept is connected to previously covered course material. (2 paragraphs)
        \item A description of the project itself. (1 paragraph)
        \item A description of the key information you want to convey to your audience (1 paragraph)
        \item An explanation of why you chose this particular medium and this particular representation of the concept, as well as a justification for why you think this mode of communication is particularly effective (1-2 paragraphs)
        \item A summary of what you learned from this project and what else you would like to learn about this topic (1 paragraph)

    \end{itemize}
\end{itemize}

\subsection*{Grading}
The entirety of the final project is worth 20\% of your final course grade. This 20\% consists of your presentation itself (scored out of 100 points) and the reflection (scored out of 40 points). Here is a rubric \footnote{Chiefly adapted from the American Astronomical Society---Chambliss award rubric.} for the \textbf{presentation}: \bigskip

\noindent
\textbf{Content: 70\%}
\begin{itemize}
%\itemsep0em
\item (35\%) Presenter introduces and describe(s) topic at level appropriate to this class [\underline{\hspace{5mm}}]
\item (40\%) Presenter explains extent of and limitations on our knowledge of the topic, including data/observations underlying knowledge [\underline{\hspace{5mm}}]
\item (20\%) Presenter provides context by drawing connections to, e.g., different areas of astronomy, concepts from lab or
lecture, other areas of science, areas outside of science, etc. [\underline{\hspace{5mm}}]
\item (5\%) Presenter chooses and cites appropriate references (i.e., goes beyond Wikipedia and popular press releases).  Presenter submits reference list. [\underline{\hspace{5mm}}]
\end{itemize} 

\noindent
\textbf{Delivery: 30\%}
\begin{itemize}
\item (35\%) Presentation has a logical flow that audience can follow [\underline{\hspace{5mm}}]
\item (25\%) Presenter can address reasonable audience questions [\underline{\hspace{5mm}}]
\item (20\%) Presentation aids (slides or board-work) or final creative project are understood by audience [\underline{\hspace{5mm}}]
\item (10\%) Presenter stays within allotted time [\underline{\hspace{5mm}}]
\item (10\%) Presenter speaks clearly, and keeps the audience engaged (with, e.g., questions, activities, etc.) [\underline{\hspace{5mm}}]
\end{itemize}
{\small [\underline{\hspace{5mm}}] = easily and concisely (4), sufficiently (3),
is somewhat able to (2), barely to did not (1)
}

\bigskip \bigskip
\noindent
The project \textbf{reflections} will be graded out of 40, with 20 points for depth/thoroughness of explanations, 10 points for clarity of writing, and 10 points for correctness.

\section{Suggested topics}

Please submit your proposed topics by \textbf{6 PM on November 13}. \bigskip
 
 \noindent
 Below is a list of potential project media as well as a non-comprehensive list of suggested topics.  You can choose something not listed, so long as it's relevant to the topics we've covered in lab (e.g., exoplanets, stars, galaxies, cosmology, etc.). If you are doing a research project, you should choose a topic that you haven't covered in depth in class or in this lab.

\medskip \noindent
More focused/specific topics often yield more compelling presentations (and are often better suited for 10-minute presentations). A sufficiently specific topic would be something like ``The Great Red Spot and other storms on Jupiter,'' while something like ``Gas giant atmospheres'' would require more specificity.

\bigskip \noindent
Example Project Ideas:
\begin{itemize}[noitemsep]
    \item A research project (with an associated PowerPoint and/or whiteboard presentation) on a topic we haven't covered in class (e.g., a new astronomy concept, an astronomer/scientist we haven’t discussed in class, an instrument/observatory/technique we haven’t discussed in class, etc.)
    \item A description of a museum exhibit that you'd design to teach the public about a specific concept
    \item A short performance or dialog
    \item A visual, audio, or mixed-media art piece (sculptures, music, \textit{etc.})
    \item A creative writing piece (e.g., poetry or a short story)
    \item Culinary arts (e.g., baked goods representing some astronomy concept, a "cookbook" on how to create stars/planets/galaxies)

\end{itemize}

\medskip \noindent
Topic Ideas:
\begin{itemize}[noitemsep]
    \item Galaxies (including our own)
        \begin{itemize}[noitemsep]
            \item Galactic dynamics (e.g., birth, growth, rotation of galaxies)
            \item Supermassive black holes
            \item Different theories of dark matter (or different dark matter candidates)
            \item The intergalactic medium (IGM)
            \item Dark matter halos and the dark matter content of different galaxies
            \item Dwarf galaxy satellites of the Milky Way
            \item Ultra-faint dwarf galaxies
            %\item Stellar life cycle---from birth to supernova!
            \item Dark energy
            \item Galaxy clusters
        \end{itemize}

    \item Stars (including our Sun)
        \begin{itemize}[noitemsep]
            \item Interior structure and chemistry of stars
            \item Asteroseismology or helioseismology 
            \item Stellar atmospheres or magnetospheres
            \item Stellar or solar winds
            \item The process of star formation (or the properties of star-forming regions in galaxies)
            %\item Stellar life cycle---from birth to supernova!
            \item Binary star systems
            \item Clusters of stars (globular clusters or open clusters)
            \item Specific types of star (e.g., T Tauri, RR Lyrae, Population III (the first stars))
        \end{itemize}

    \item (Exo)Planets
        \begin{itemize}[noitemsep]
            \item Solar system formation and history
            \item Planet X
            \item Proto-planetary disks
            \item Planet and planetesimal formation
            \item Brown dwarfs
            \item Exoplanet detection methods not discussed in class (e.g., microlensing, astrometry)
            \item Exoplanet atmospheres
        \end{itemize}
    
    \item Astrobiology
    \begin{itemize}[noitemsep]
        \item The Search for Extraterrestrial Life (SETI)
        \item The Drake equation
        \item Dyson spheres (or other hypothetical megastructures)
        \item Technosignatures vs. Biosignatures
        \item Communication and signal detection; candidate SETI signals
        \item Breakthrough Listen or Breakthrough Starshot
    \end{itemize}
    
    \item Telescopes and spacecrafts
        \begin{itemize}[noitemsep]
            \item Specific missions/projects (e.g., Hubble Space Telescope, James Webb Space Telescope, Kepler, TESS, Nancy Grace Roman Space Telescope, Vera C. Rubin Observatory, Thirty Meter Telescope).
            \item Astronomy at specific wavelengths (e.g., Radio astronomy and very-long-baseline interferometry (VLBI), sub-millimeter astronomy, X-ray astronomy, gamma-ray astronomy)
            \item NASA budget, missions, proposals (i.e., how funding decisions are made)
            \item Space policy (i.e., laws governing space)
        \end{itemize}

    \item Controversial Astronomy
        \begin{itemize}[noitemsep]
            \item Planet X
            \item Extraterrestrial life 
            \item Phosphine on Venus
        \end{itemize}

    \item Miscellaneous
        \begin{itemize}[noitemsep]
            \item The Big Bang and the early Universe (e.g., inflation, nucleosynthesis, the epoch of recombination, the epoch of reionization)
            \item The cosmic microwave background (CMB)
            \item Gravitational waves and LIGO
            \item Compact objects (Black holes, neutron stars, pulsars, magnetars, white dwarfs)
            \item High-energy explosions (Fast Radio Bursts or Gamma-Ray Bursts)
            \item A biographical presentation on a famous astronomer. If you do this, choose 1-2 scientific contributions to emphasize. Some suggestions for scientists:
            \begin{itemize}[noitemsep]
                \item Annie Jump Cannon (spectra of stars)
                \item Cecilia Payne-Gaposchkin (the composition of stars)
                \item Vera Rubin (dark matter)
                \item Jocelyn Bell Burnell (radio pulsars)
                \item Nancy Grace Roman (stellar classification and motion)
                \item Jill Tarter (SETI)
                \item Sara Seager (exoplanets)
                \item Caroline Herschel (comets)
                \item Annie Maunder (sunspots, solar corona, eclipses)
                \item Margaret Kivelson (solar wind, Europa’s ocean)
                
            \end{itemize}
            
            \item A recent or historically significant astronomy paper (I recommend searching through \url{https://ui.adsabs.harvard.edu/} or \url{https://arxiv.org/archive/astro-ph}, or asking me for help finding a paper).
        \end{itemize}
        
\end{itemize}

%Other options (equally encouraged):
%\begin{itemize}    
    
    %I recommend looking at the Daily Paper
%        Summaries on Astrobites (\url{https://astrobites.org/}).  This is a
%        blog that summarizes scientific papers at an introductory level;
%        summaries are written by astro graduate students and aimed for
%        undergrad/grad students alike.
%        Other sources of brief, accessible scientific papers include Nature
%        (\url{https://www.nature.com/}), Nature Astronomy
%        (\url{https://www.nature.com/natastron/}), and
%        Science (\url{https://www.sciencemag.org/}).
%        For popular press that can direct you to interesting papers, consider:
%        \url{https://www.quantamagazine.org/physics} or
%        \url{https://www.scientificamerican.com}.

%    \item Biographical study of a famous astronomer or planetary scientist.
%        If you do this, choose at least one scientific contribution to
%        emphasize.
%        \begin{itemize}[noitemsep]
%            \item Galileo, Kepler, etc.
%            \item Caroline Herschel (comets)
%            \item Annie Maunder (sunspots, solar corona, eclipses)
%            \item Annie Jump Cannon (spectra of stars)
%            \item Cecilia Payne-Gaposchkin (the composition of stars)
%            \item David Jewitt (trans-Neptunian objects, comets)
%            \item Margaret Kivelson (solar wind, Europa's ocean)
%            \item Carl Sagan (science communication; solar system,
%                astrobiology)
%            \item Jill Tarter (SETI)
%            \item Sara Seager (exoplanets)
%        \end{itemize}
%\end{itemize}

\end{document}



\section{Suggested Topics}

\end{document}