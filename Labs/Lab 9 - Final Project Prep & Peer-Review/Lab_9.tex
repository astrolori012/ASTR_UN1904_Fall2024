\documentclass[11pt]{article}% uses letterpaper by default

%---------- Uncomment one of them ------------------------------
\usepackage[includeheadfoot, top=1in, bottom=1in, hmargin=1in]{geometry}

% \usepackage[a5paper, landscape, twocolumn, twoside,
%    left=2cm, hmarginratio=2:1, includemp, marginparwidth=43pt, 
%    bottom=1cm, foot=.7cm, includefoot, textheight=11cm, heightrounded,
%    columnsep=1cm, dvips,  verbose]{geometry}
%---------------------------------------------------------------
\usepackage{fancyhdr}
\renewcommand{\footrulewidth}{0.4pt}% default is 0pt
\usepackage{verbatim}
\usepackage{url}
\usepackage{cancel}
\pagestyle{fancy}
\usepackage{graphicx}
\usepackage{setspace}
\singlespacing
%\doublespacing
%\onehalfspacing
\usepackage{varwidth}

\newcommand{\degrees}{\ensuremath{^\circ}}
\newcommand{\arcmin}{\ensuremath{'}}
\newcommand{\arcsec}{\ensuremath{"}}
\newcommand{\hours}{\ensuremath{^\mathrm{h}}}
\newcommand{\minutes}{\ensuremath{^\mathrm{m}}}
\newcommand{\seconds}{\ensuremath{^\mathrm{s}}}

\newcommand{\s}[0]{\phantom{i}} %sets up \s command
\newcommand{\m}[0]{\phantom{abcde}} %sets up \m command
\providecommand{\e}[1]{\ensuremath{\times 10^{#1}}} %sets up \e command
\setlength{\parindent}{0.2in} %new paragraph indent
\usepackage{indentfirst} % indent the first paragraph of a section
\usepackage{amsmath,amssymb}
\usepackage{enumitem}

\lhead{Astronomy Lab II}
\rhead{Fall 2024}
\lfoot{Porter}
\rfoot{Wed 6-9pm}
\cfoot{\thepage}

%\newcommand{\exercisename}{7}
\begin{document}

\begin{center}
\huge{Lab 9: Final Projects and Peer Review}\\ \medskip \Large{November 13, 2024}
\end{center}

%%%%%%%%%%%%%%%%%%%%%%% INTRODUCTION %%%%%%%%%%%%%%%%%%%%%%%
\section{Final Projects}

By now, everyone should have submitted their final project ideas. Take the first 2 hours of class, until 8PM, to work on developing your project and make progress.

By 7:45PM, please compile your progress and work into something that can be shared with another student, such as a document, slideshow, \textit{etc.}

\section{Peer-Review}

At 8 PM, now that you have some (hopefully substantial) work completed on your project, you \textbf{will need to group into pairs.} Please choose someone you haven't worked with often - I will check! This will ensure you get feedback from someone who doesn't know you quite as well, and may be able to spot mistakes easier!

\subsection{Leaving Feedback for Someone Else}

Exchange work-in-progress projects with your partner. Independently, begin reviewing each others' projects. You should be focusing on the overall content, grammar, presentation, design, sources, \textit{etc.} The following questions will aid in your review. 

\textbf{Please keep in mind that you should be distributing this feedback to your partner in order to help them improve their own project.} This means your answers should be honest (no one should have flawless projects at this point), but any criticism should be polite and constructive.

\textbf{Log your answers in your lab write-up. These answers should be \textit{detailed} and \textit{thorough}. Answers that are a few words long are not helpful to your partner.}


\begin{enumerate}
\setcounter{enumi}{0}
    \item List the name of your partner.

    \item What form will your partner's project be in? For example, are they creating a presentation/slideshow or are they doing a creative option?

    \item From their current work, what do you think the focus of their project is about? Does this match well with the topic they submitted?

    \item What are some of the key points this person uses in their projects, and how are they conveyed?

    \item Is any context provided? For example, how does the author convey the meaning and importance of their project, or is it unclear?

    \item What sources are used and cited? Do they appear to be reliable and scientific?

    \item Is the content explained at a level appropriate for the class (i.e., using some depth beyond surface-level topics, but can be understood by a general audience)? Is there anything too broad, obvious, or confusing?

    \item Do you notice any distracting mistakes? If so, what are they? 

    \item How does this person incorporate creativity and critical thinking into their project? If they don't at all, list why you think this.

    \item Does presented information appear to be factually accurate? Why or why not?

    \item How is the project organized? Does the structure make sense for the information they are trying to convey?

    \item Did you think this was an engaging and/or unique presentation? 

    \item How does the project's design compliment the topic, if at all? How can this be improved to create a memorable. interesting, and educational project?

    \item Please list any additional feedback for the individual who you reviewed here. This can be in any area, but should be with the intent of helping to improve their final project and reflection. 

    \item Do you have any final questions for the person you reviewed?

\end{enumerate}

\subsection{Reviewing Your Own Feedback}

Please begin reviewing your own feedback.

\begin{enumerate}
\setcounter{enumi}{15}
    \item Write the name of your reviewer here.

    \item Have you decided to change any components or directions of your final project after receiving your peer feedback? If so, describe how and why. This answer should be detailed and thorough to show how the peer-review did or did not change your plans.

    \item Did you notice any particular challenges when reviewing someone else's in-progress project? Typically, in astronomy, peer-review covers research that is presented in a final or near-final format; what advantages or disadvantages exist to in-progress review versus final review?

    \item Did you disagree with any of your feedback? If so, why?

\end{enumerate}

\end{document}

