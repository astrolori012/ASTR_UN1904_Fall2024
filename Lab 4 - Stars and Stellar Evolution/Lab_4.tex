\documentclass[11pt]{article}
\usepackage[includeheadfoot, top=1.0in, bottom=1.0in, hmargin=1.0in]{geometry}
\usepackage[utf8]{inputenc}
\usepackage{fancyhdr}
\pagestyle{fancy}
\usepackage{setspace}
\usepackage{tabularx}
\usepackage{xcolor}
\usepackage{cancel}
\usepackage{amsmath,amsfonts}
\usepackage{graphicx}
\usepackage{siunitx}
\usepackage{amssymb}

\usepackage[hyphens]{url}
\usepackage{hyperref}
\usepackage{enumitem}

\lhead{Astronomy Lab II}
\rhead{Fall 2024}
\lfoot{Porter}
\rfoot{Wed 6-9pm}
\cfoot{\thepage}

\begin{document}

\begin{center}
\huge{Lab 4: Stars \& Stellar Evolution}\\ \medskip \Large{October 2, 2024}
\end{center}

\section{Introduction: Reaching for the Stars}

Out of cold and dense clumps of gas, stars are born. As they fuse elements in their fiery cores, stars age and evolve. And, after millions ($10^6$) to billions ($10^9$) of years, these stars die -- either collapsing into a compact stellar remnant or exploding in a brilliant supernova. This same material that is expelled when a star meets its end is the \emph{same} material out of which the next generation of stars will arise -- baby stars form out of dead stars, sustaining a seemingly endless stellar life cycle.  

The study of stellar evolution requires detailed knowledge of chemistry, fluid dynamics, nuclear physics, and more, making it an active area of research. However, the systematic study of stellar evolution is only about a hundred years old, beginning in the early 1910s with the work of Ejnar Hertzsprung and Henry Norris Russell. Hertzsprung and Russell each discovered that, given a group of stars, plotting the \textbf{brightness} against the \textbf{surface temperature} of each star reveals clear trends linked to the steps in the stellar life cycle. This plot of stellar brightness vs. temperature, called the \textbf{\emph{Hertzsprung-Russell diagram}} (or the H-R diagram), still serves as an incredibly useful visual and organizational aid today.

In this lab, you will construct your own H-R diagram to reveal connections between the brightness, temperature, color, mass, size, and chemical composition of stars. You will also explore how different stars evolve along the H-R diagram as they age and how we can identify stars at different stages of their life cycles. The amount of information you can extract from the H-R diagram will certainly leave you \emph{star}struck!

\section{Building an H-R Diagram}
Before looking at any data or making any plots, let's make some hypotheses. \textbf{Record your responses in your lab write-up}:
\begin{enumerate}
    \item What trends (if any) do you expect to emerge when we plot a group of stars on a brightness vs. temperature plot? Beyond just brightness and temperature, think about how stellar mass, size, color, and chemical makeup might affect a star's position on the plot.
    
    \item How do you expect the H-R diagram to change when we plot brightness vs. \emph{color} instead of brightness vs. temperature? (hint: stars can be well-approximated as black bodies! Use what we've learned about color and stars in the last few labs...) 
    
\end{enumerate}

\subsection{The brightest stars}
% explain magnitude, luminosity, color. how are mag and luminosity related to brightness?

Open up the spreadsheet named ``stars.xlsx'' in the ``Lab 4'' folder under the ``Files'' tab on CourseWorks. The first tab of the spreadsheet (labeled ``brightest'') contains data for the 34 brightest stars in the night sky. Using this data, \textbf{complete the following, recording your responses in your lab write-up}:
\begin{enumerate}
    \setcounter{enumi}{2}
    
    \item An H-R diagram plots \emph{brightness} vs. \emph{surface temperature} (i.e., the temperature of the outer layer of a star). However, there are multiple ways of quantifying brightness. The \textbf{luminosity} expresses the intrinsic energy output of a star -- effectively, how much light is produced per second. On the other hand, the \textbf{magnitude}, a unitless quantity, expresses brightness on an inverse logarithmic scale (where, confusingly, \emph{smaller} or \emph{more negative} magnitudes correspond to \emph{brighter} stars).
    
     Construct an H-R diagram for these 34 bright stars by making a scatter plot with the \emph{absolute magnitude} on the vertical axis (see column C) on the vertical axis and the surface temperature on the horizontal axis. For historical reasons, the temperature should \emph{increase} as we move to the \emph{left} along the diagram, so make sure to \emph{invert} your horizontal axis. See the 'Excel Guide' on Courseworks if you need help. You should also invert the y-axis, since we want brightness to \emph{increase} on the vertical axis. Make sure your axes are labeled with the correct units!

    
    \item On your magnitude vs. temperature H-R diagram from above, do you see any groups or clumps of stars? If so, describe where on the diagram these groups lie, and (if possible) circle or box these groups on your diagram.
    
    \item Where does the Sun lie on your diagram? Compared to the rest of the stars, is the Sun brighter or dimmer than average? Is the Sun hotter or colder than average? You can find the properties of the Sun in the first row of data.
    
    \item Do you think the H-R diagram you've plotted for these bright stars is representative of all stars? Why or why not?
    
\end{enumerate}

\subsection{The closest stars}
Now, go to the second tab in ``stars.xlsx,'' labeled ``nearest.'' This sheet contains data for the 30 closest stars to Earth. \textbf{Complete the following in your lab write-up}:
\begin{enumerate}
\setcounter{enumi}{6}

    \item \textbf{Before looking at any new plots}, consider the \emph{biases} in the two data sets we've considered so far (i.e., the brightest stars and the closest stars). Whenever you're working with any data set (astronomical or otherwise), it's imperative that you always consider how the data may be biased!
    \begin{enumerate}
        \item Do you expect your H-R diagram of the \emph{closest} stars to differ from your H-R diagram of the \emph{brightest} stars? Why and how?
        
        \item Do you expect your H-R diagram of the \emph{closest} stars to differ from that of the \emph{general population} of stars? Why and how?
    \end{enumerate}
    
    \item Now, construct an H-R diagram for the 30 closest stars by again making a scatter plot with the (reversed) magnitude on the vertical axis and the surface temperature on the horizontal axis. Once again, remember to invert your horizontal axis so that temperature is greatest to the \emph{left} of the diagram.
    
    \item How does your H-R diagram for the closest stars differ from your H-R diagram for the brightest stars? This answer can be a few words if you want.
    
    \item How does the Sun compare to other nearby stars (in terms of, for example, brightness and temperature)? This answer can be a few words if you want.
    
\end{enumerate}

\subsection{The full diagram}
You should have seen a slight difference between your two H-R diagrams, with the diagram of the brightest stars having more points towards the top right and top left of the plot, while the diagram of the nearest stars should have had more points closer to the Sun and towards the bottom right of the diagram. Now, let's combine these two populations onto a single diagram. \textbf{Complete the following in your lab write-up}:
\begin{enumerate}
\setcounter{enumi}{10}

    \item Go to the ``combined'' tab on the ``stars.xlsx'' spreadsheet and once again make a scatter plot of magnitude vs. surface temperature.
    
    \item Stars spend most of their lives on the so-called \emph{main sequence}. Main-sequence stars span a wide range of magnitudes and temperatures.
    \begin{enumerate}
        \item Where is the main sequence on your H-R diagram? Circle it in \textbf{black} in your lab write-up, or describe it in words if you're unable to do so. 
        
        \item Not all stars lie on the main sequence -- \emph{giant} stars are brighter and (on average) cooler than main sequence stars. Where is the giant branch on your H-R diagram? Circle it in \textbf{\textcolor{red}{red}} in your lab write-up, or describe it in words if you're unable to do so. 
    
    \end{enumerate}
    
    \item We're still missing a key component of the H-R diagram: the \emph{white dwarfs}.
    \begin{enumerate}
        \item Using data from the ``complete'' tab of the ``stars.xlsx'' spreadsheet, plot an H-R diagram complete with a main sequence, a giant branch, and a population of white dwarfs.
        
        \item Compare your new diagram to your previous diagrams to help identify the white dwarfs. Circle these stars in \textbf{\textcolor{blue}{blue}} in your lab write-up, or describe in a couple words where the white dwarfs are located. 

    \end{enumerate}
    
    
\end{enumerate} 

\section{Stellar Evolution}
We've already talked a little about how the structure of the H-R diagram hints at the underlying process of stellar evolution. Now, let's see this in action. Navigate to \url{https://starinabox.lco.global/}; this ``Star in a Box'' animates stellar evolution along the H-R diagram. Click ``Open the Lid'' to see the H-R diagram. Use this to \textbf{answer the following questions in your lab write-up}:
\begin{enumerate}
\setcounter{enumi}{13}

    \item Using the drop-down menu in the bottom left corner, progressively vary the initial mass of the star from 0.2 solar masses to 40 solar masses. The black dot on the H-R diagram denotes the star's initial position on the main sequence. How does varying the mass change the star's position along the main sequence? 
    
    \item To watch a star evolve, select an initial mass and click the play button (the right-arrow in the lower right corner); if the animation is too fast, you can adjust the animation speed with the drop-down menu to the right of the play button.
    \begin{enumerate}
    
        \item Describe the phases of evolution of a Sun-like star (a star with an initial mass of 1 solar mass) in the H-R diagram. What do you think is happening to the star (in terms of radius and temperature) as it traverses the nearly-horizontal stretch towards the top of its evolutionary track, where the luminosity is staying the same? (\emph{hint}: for a fixed luminosity, a star's radius and temperature vary inversely)  
        
        \item In the previous section, you discovered the main sequence, the giant branch, and the white dwarfs by plotting a collection of stars on the H-R diagram. How are these three populations connected via stellar evolution? (\emph{hint}: think about how a star gets its energy, and at what point in a star's lifetime the main sequence, giant branch, and white dwarfs happen)
        
        \item You should find that there are three possible ways in which a star can complete its life. What are these three end-of-life products? Check the upper-right corner. At which \emph{initial} stellar masses do we transition from one end product to another? This answer can be a list.
        
    \end{enumerate}

\end{enumerate}



\section{Wrapping Things Up}
\textbf{Answer the following questions in your lab write-up}:
\begin{enumerate}
\setcounter{enumi}{15}

    \item Reflect a bit on the H-R diagram:
    \begin{enumerate}
        \item In a sentence, describe what an H-R diagram is.
        
        \item Describe the major groups of stars that show up on an H-R diagram. 
        \begin{enumerate}
            \item How do the properties of a typical star in each of these groups compare to those of our Sun?
        \end{enumerate} 
        
        \item How does the process of stellar evolution connect to the H-R diagram?
        
    \end{enumerate}
    
\end{enumerate}


\end{document}