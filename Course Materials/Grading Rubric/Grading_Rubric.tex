\documentclass[10pt]{article} 

\usepackage[includeheadfoot, top=1in, bottom=1in, hmargin=1in]{geometry}

%\usepackage{amsmath, amsfonts, amssymb,epsfig,graphicx}

\usepackage{fancyhdr}
\usepackage{url}
\pagestyle{fancy}
\usepackage{setspace}


%\doublespacing
\singlespacing
%\onehalfspacing

\lhead{Astronomy Lab II}
\rhead{Fall 2024}
\lfoot{L. Porter}
\rfoot{Wed 6-9pm}
\cfoot{\thepage}
\rfoot{}

\begin{document}

\begin{center}
\LARGE{ASTR1904: Lab II -- Section 001} \\ \medskip \Large{Grading Rubric}\\ 
\end{center}

%%%%%%%%%%%%%%%%%%%%%%% WRITE-UP %%%%%%%%%%%%%%%%%%%%%%%
\section*{Detailed Lab Write-Up Guidelines}

The primary goal of this class is to teach you how science is actually done. This means keeping a record of everything and using logic to reason what you do, why you do it, and how. Scientists can \textit{rarely} make assumptions or choices without defending and justifying why they made those specific choices! State any assumptions (even if we covered it in class), show work for calculations (don't skip steps), use correct units, use complete sentences, \textit{etc.} You will not be graded on perfect English grammar, but I should be able to follow what you are writing. \\ 

You will complete labs in groups of 2 or 3, with an exception for observing labs, but make sure to keep your own records and complete your own write-up. You are welcome to work together, but each group member needs to submit their own \textbf{unique} write-up; copying work, even in groups, is plagiarism. I should be able to reproduce your answers with just the information in your write-up. Below are some formatting and specific content requests. \\

\begin{itemize}
\item Begin each lab write-up as a new document and have your name, names of collaborators (except for observing labs), the lab title, and the date at the top.
\item \textit{Always} include units where necessary, and always label plot axes.
\item Written answers should be in complete sentences. Please do not submit one or two-word answers, or sentence fragments, unless the question specifically calls for it. This is essential to the course objective of learning to effectively communicate science and results, as well as understand them.
\item Show \textbf{all} steps in your calculations, including if you need to do any unit conversions. If you make a mistake, I should be able to see where it happened. This will allow me to give you as many points as possible if your answer is incorrect, as opposed to if you simply state an incorrect answer.
\item Put a box around (or highlight) numerical answers.
\item If you don't know the answer to a question or if you know your answer is wrong but can't figure out why:
\subitem -Talk to me! I am happy to help.
\subitem -Tell me what you do know: what steps did you take to arrive at your answer? What do you know about the problem in general, and how do you know your answer is wrong? For example, you might calculate that the Sun is 3 meters from Earth; this is a red flag that something happened in your calculation. Where do you think you might have gone wrong? 
\subitem -Even if your answer is incorrect, a good scientist is able to speculate about where they went wrong. I will \textbf{not} penalize you for being able to identify your own mistakes, and you will earn more points than by just submitting an incorrect answer.   
\end{itemize}

%%%%%%%%%%%%%%%%%%%%%%% RUBRIC %%%%%%%%%%%%%%%%%%%%%%%
\section*{Grading Rubric}

Each lab write-up will be weighted equally. Since your 9 highest lab grades (your lowest will automatically be dropped) will be counted towards 65\% of your final grade, each lab is worth \textit{about} 7.22\% of your final grade. Points will be assigned, depending on the type of question, based on three categories:

\begin{enumerate}
\item \textbf{Clarity of writing:} (Maximum 4 points)
\begin{itemize}
    \item 0 points: Answer is not present or question is left blank.
	\item 1 point: Little justification of answers, hard to follow explanations or logic. Writing is not in complete sentences, such as sentence fragments.
	\item 2-3 points: Explanations need additional work or clarification may be in order, scientific terms are used appropriately most of the time. 
	\item 4 points: Explanations are clear and every answer is fully justified, scientific terms are used correctly, and student clearly shows a high level of effort and engagement with the lab material.
\end{itemize}
\item \textbf{Quantitative Reasoning:} (Maximum 3 points)
\begin{itemize}
    \item 0 points: Answer is not present or question is left blank.
	\item 1 point: Graphs, diagrams or equations are misinterpreted or not well-explained in the context of the lab, not included, or incorrect.  Units are incorrect and/or misused.  Quantitative values used in arguments, if present, are incorrect and/or unreasonable in the context they are used.  
	\item 2 points: Graphs, diagrams or equations may be misinterpreted but the explanation may be valid in the context of the lab, and a good-faith effort is evident.  Units are incorrect but the correct dimension (distance or speed, for example) is used. A few key assumptions may be missing or additional reasoning is needed to reach the final and correct answer, but student show some understanding of the data/equation/diagram and what the question is asking.
	\item 3 points: Graphs, diagrams or equations are correctly interpreted, used appropriately, and well-explained.  Units are consistent and correct.  Quantitative values used in arguments are correct and used appropriately in the context of the argument.
\end{itemize}
\item \textbf{Correctness of answers:} (Maximum 3 points)
\begin{itemize}
    \item 0 points: Answer is not present or question is left blank.
	\item 1 point: The answers are incorrect and off by many orders of magnitude, in a way that should have made it clear that the answer is incorrect. Student does not identify that their answer is significantly off, nor do they provide any physical reasons on how they know/what might have happened.
	\item 2 points: The answers are incorrect, but could be reasonable given the question. There is not sufficient work shown to identify the error. Alternatively, the answers are incorrect and off by many orders of magnitude, but are identified as such and a hypothesis is made as to why they are off by so much, including physical reasons from their knowledge of the lab topic.
    \item 2.5 points: The answers are incorrect, but work is shown, and there is a clear indication of understanding of the problem at hand.  Errors are extremely minor (calculator errors, forgetting to carry a decimal or exponent, \textit{etc}, that would result in a correct answer otherwise. 
	\item 3 points: The answers are correct to the degree of accuracy specified by the specific question.
\end{itemize}
\end{enumerate}

For example, if a lab question asks you to complete a calculation and explain what it means, this question will be graded on 1. Clarity of Writing, 2. Quantitative Reasoning, and 3. Correctness. \\

If a question, such as at the end of the lab, asks for a subjective answer or your thoughts on a topic, the question will be graded on 1. Clarity of Writing. \\

Calculation-only answers, or those only requiring a graph (with no explanation), will be graded on 2. Quantitative Reasoning and 3. Correctness of answers. 


\end{document}